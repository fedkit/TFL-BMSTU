\documentclass[a4paper,12pt]{article}

\usepackage{graphicx}
\usepackage{amsmath}
\usepackage{amssymb}
\usepackage[russian]{babel}
\usepackage[utf8]{inputenc}
\usepackage[T2A]{fontenc}
\usepackage{geometry}
\geometry{top=2cm, bottom=2cm, left=2cm, right=2cm}

\usepackage{indentfirst}
\setlength{\parindent}{0.75cm}

\begin{document}

\begin{center}
    {\Large \textbf{Контрольная работа №2}}\\[6pt]
    \includegraphics[width=0.85\textwidth]{rk2.png} 
\end{center}

\vspace{-0.5cm}

\begin{center}
\textbf{Задание 1}
\end{center}

Слова образованы с помощью грамматики 
$S \rightarrow bSaS \mid aSbS \mid a$ ,
и в них максимальный отрезок из букв $b$ длиннее максимального отрезка из букв $a$.


Рассмотрим семейство слов языка, получающиеся алгоритмом
$b\underline{S}aS \rightarrow bb\underline{S}aSaS \xrightarrow{} bbb\underline{S}aSaSaS \xrightarrow{} \dots$

Изначально слово имеет одну букву b, одну буквы a и два нетерминала S. После каждого раскрытия S общее количество a, b и S увеличивается на 1.

Рассмотрим слова у которых по итогу количество букв b в начале будет нечетно.

\[
b^{2p+1}(Sa)^{2p+1}S.
\]

Преобразуем выражение:
\[
b^{2p+1}(Sa)^{2p+1}S
= b^{2p+1}(Sa)^p Sa (Sa)^p S.
\]

Далее применяем правила грамматики:
\[
b^{2p+1}(Sa)^p Sa (Sa)^p S
\;\Rightarrow\;
b^{2p+1} a^{2p} b (Sa)^p Sa Sa S
\]

\[
\Rightarrow\;
b^{2p+1} a^{2p} b a^{2p} b Sa Sa Sa S
\]

\[
\Rightarrow\;
b^{2p+1} a^{2p} b a^{2p} b a^{7}.
\]

Таким образом, слово
\[
b^{2p+1} a^{2p} b a^{2p} b a^{7}
\]
принадлежит рассматриваемому языку при \( p \ge 4 \).

По лемме о накачке для контекстно-свободных языков существует \(p_1\), удовлетворяющее условиям леммы. Рассмотрим \(n > p_1\) и слово
\[
w = b^{2n+1} a^{2n} b a^{2n} b a^{7}.
\]

Разобьём слово на шесть частей:
\[
w =
\underbrace{b^{2n+1}}_{q_1}
\underbrace{a^{2n}}_{q_2}
\underbrace{b}_{q_3}
\underbrace{a^{2n}}_{q_4}
\underbrace{b}_{q_5}
\underbrace{a^{7}}_{q_6}.
\]

Рассмотрим нетревиальные положения накачиваемой подстроки \(vxy\):

\begin{itemize}
    \item \(q_1\): при \(i=0\) длина блока из \(b\) уменьшается, и слово перестаёт принадлежать языку.
    \item \(q_2\): при \(i \to \infty\) длина подстроки из \(a\) больше чем из \(b\).
    \item \(q_3\): при \(i=0\) получаем блок \(a^{4n}\), что невозможно по определению языка.
    \item \(q_4\): аналогично случаю \(q_2\).
    \item \(q_5\): при \(i=0\) остаётся суффикс \(a^{2n+7}\), что не допускается.
    \item \(q_6\): при \(i \to \infty\) длина подстроки из \(a\) больше чем из \(b\).
\end{itemize}

Рассмотрим случаи, когда \(vxy\) пересекает границы частей:

\begin{itemize}
    \item \(q_1, q_2\): при \(i=0\) длина подстроки из \(b\) становится не больше длины подстроки из \(a\), что противоречит свойствам языка.
    \item \(q_2, q_3\): при \(i=0\) и \(i \to \infty\) нарушается структура слова или длина подстроки из \(a\) больше чем из \(b\).
    \item \(q_3, q_4\): аналогично предыдущему случаю.
    \item \(q_4, q_5\): аналогично предыдущему случаю.
    \item \(q_5, q_6\): при \(i \to \infty\) слово не принадлежит языку.
    \item \(q_2, q_3, q_4\): при \(i \to \infty\) количество подряд идущих букв \(a\) становится больше количества подряд идущих букв \(b\).
    \item \(q_4, q_5, q_6\): при \(i \to \infty\) количество подряд идущих букв \(a\) становится больше количества подряд идущих букв \(b\).
\end{itemize}

Прочие тривиальные случаи также не удовлетворяют условиям леммы о накачке. Следовательно, данный язык не является контекстно-свободным.

\begin{center}
\textbf{Задание 2}
\end{center}

\[
L = \{\, a^n b^n c^k \mid n \neq m \;\vee\; k > n + m \,\}.
\]

Возьмём дополнение языка:
\[
L' = \{\, a^n b^n c^k \mid 
\underbrace{n = m}_{(1)} \;\wedge\; 
\underbrace{k \le n + m}_{(2)} \,\}.
\]

Пусть по лемме о накачке для контекстно-свободных языков существует число \(p\), удовлетворяющее условиям леммы. Рассмотрим \(n > p\) и слово
\[
w =
\underbrace{a^n}_{q_1}
\underbrace{b^n}_{q_2}
\underbrace{c^{2n}}_{q_3}.
\]

Рассмотрим возможные случаи накачки подстроки \(vxy\):

\begin{itemize}
    \item \(q_1\): при любом \(i \neq 1\) слово перестаёт удовлетворять правилу (1).
    \item \(q_2\): аналогично случаю \(q_1\).
    \item \(q_3\): при \(i \to \infty\) слово нарушает правило (2).
    \item \(q_1, q_2\): либо слово перестанет принадлежать языку по своей структуре, либо при \(i=0\) нарушается правило (2).
    \item \(q_2, q_3\): при \(i \to \infty\) нарушается правило (2).
\end{itemize}

Следовательно, дополнение языка \(L'\) не является контекстно-свободным, а значит исходный язык \(L\) не DCFL.

\begin{center}
\textbf{Задание 3}
\end{center}
\begin{center}
\[
\begin{array}{rcll}
S & \to & ASA & ; \quad A_1.b > A_2.b, \; S_1.b > A_2.b, \; S_0.b := S_1.b + 2 \cdot A_1.b \\
S & \to & b & ; \quad S.b := 1 \\
A & \to & aA & ; \quad A_0.b := A_1.b \\
A & \to & bbA & ; \quad A_0.b := A_1.b + 2 \\
A & \to & \varepsilon & ; \quad A.b := 0
\end{array}
\]
\end{center}

Проанализируем слова, принадлежащие языку. Фактически слова разбиваются на две части, где разделителем служит буква \(b\). В левой части количество \(b\) больше, чем в правой, а общее количество букв \(b\) нечётное. Атрибуты грамматики отвечают за подсчёт количества букв \(b\).

Пересечём язык с регулярным выражением \((bb)^* b a (bb)^*\). Тогда получаем язык
\[
L = \{\, b^n a b^m \mid n > m, \; n \text{ нечёт}, \; m \text{ чёт} \;\text{+ еще условия}\,\}.
\]

Так как есть условие \(A_1.b > A_2.b\), выясним, при каких значениях \(m\) возможно минимальное значение \(n\):

\begin{itemize}
    \item \(m = 0\): \(n = 3\), слово \(bbb a\) принадлежит \(L\)
    \item \(m = 2\): \(n \neq 5\), \(n = 7\), слово \(bbbbbbb ab b\) принадлежит \(L\), так как при \(n = 5\) невозможно разбиение слова без нарушения условия грамматики
    \item \(m = 4\): \(n = 9\)
    \item \(m = 6\): \(n \neq 11\), минимальное \(n = 13\), аналогично при \(m = 2\)
    \item \(m = 8\): \(n = 15\)
    \item \(m = 10\): \(n = 17\)
    \item \(m = 12\): \(n = 19\)
    \item \(m = 14\): \(n = 21\)
\end{itemize}

Рассмотрим последовательности слов:

\[
\begin{aligned}
& b^3 a \in L, && (n,m)=(3,0); \quad (n,m) \neq (5,2) \implies (n,m)=(7,2) \\
& b^6 \,(b^3 a) \, b^4 \in L, && (n,m)=(9,4); \quad (n,m) \neq (11,6) \implies (n,m)=(13,6) \\
& b^{18} \,(b^6 (b^3 a) b^4) \, b^{16} \in L, && (n,m)=(27,20); \quad (n,m) \neq (29,22) \implies (n,m)=(31,22) \\
& b^{54} \,(b^{18} (b^6 (b^3 a) b^4) b^{16}) \, b^{52} \in L, && (n,m)=(81,72); \quad (n,m) \neq (83,74) \implies (n,m)=(85,74) \\
& \dots
\end{aligned}
\]

В парах \((n,m)\), указанных в неравенствах, не существует сценария, при котором не нарушается правило
\[
S_1.b > A_2.b \quad \text{при} \quad S_0.b := S_1.b + 2 \cdot A_1.b.
\]

Можно заметить, что количество букв \(b\) слева удовлетворяет \(n \neq 3^l + 2\), где \(l \ge 1\). Как только \(n = 3^l\), следующее возможное значение \(n = 3^l + 4\), и разность \(n-m\) увеличивается на 2.  

Таким образом, для \(n \in [3^l + 4, 3^{l+1}]\) имеем:
\[
n-m = 2l + 3; \quad m = n - 2l - 3.
\]

Введём функцию \(f(x)\), которая для натуральных нечётных чисел \(x \ge 3\) находит подходящее \(l\) через промежуток \([3^l+4,3^{l+1}]\) и вычисляет \(m = f(x)\). Тогда слова языка можно строить как
\[
b^n a b^{f(n)}.
\]

{Применим лемму о накачке.} Пусть существует число \(p\), удовлетворяющее условию леммы. Рассмотрим слова
\[
w = b^n a b^{f(n)}, \quad \text{где } f(n) > p.
\]

Разобьём слово на три части для применения леммы о накачке:
\[
w = \underbrace{b^n}_{q_1} a \underbrace{b^{f(n)}}_{q_2}.
\]

\begin{itemize}
    \item \(q_1\): при \(i=0\) уменьшается количество букв \(b\) слева, n меньше быть не может, если зафиксировать f(n).
    \item \(q_2\): при \(i \to \infty\) получаем \(f(n) > n\), что невозможно.
    \item \(q_1, a\): нарушается структура слова или количество a > 1.
    \item \(a, q_2\): нарушается структура слова или количество a > 1.
    \item \(q_1, a, q_3\): если \(v\) или \(y\) в \(vxy\) содержат букву \(a\), то портится структура слова; если же \(v \subseteq q_1\) и \(y \subseteq q_3\), тогда для определенного \(i\) получаем
    \[
    f(n') \neq m',
    \]
    где \(n'\) и \(m'\) — новое количество букв \(b\) слева и справа. Это связано с тем, что количество букв \(b\) растёт линейно (если накачивается одна буква) или экспоненциально, а функция \(f(x)\) кусочно-линейная.
\end{itemize}

Следовательно, такие слова невозможно накачать, и по лемме о накачке язык \(L\) не является контекстно-свободным. А значит, исходный язык также не контекстно-свободный.

\end{document}

