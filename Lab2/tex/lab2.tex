\documentclass[a4paper,12pt]{article}

\usepackage{graphicx}
\usepackage{amsmath}
\usepackage{amssymb}
\usepackage[russian]{babel}
\usepackage[utf8]{inputenc}
\usepackage[T2A]{fontenc}
\usepackage{geometry}
\geometry{top=2cm, bottom=2cm, left=2cm, right=2cm}

\usepackage{indentfirst}
\setlength{\parindent}{0.75cm}

\begin{document}

\begin{center}
    {\Large \textbf{Лабораторная работа №2}}\\[6pt]
    {\large Вариант 14}
\end{center}

\vspace{-0.5cm}

\[
\left( aa^{*}ab \;\middle|\; bbabb \;\middle|\; ab\,b^{*}abab \right)^{*}
\, baba \,
(a \mid b)(a \mid b)
\left( (aa)^{*}ab \;\middle|\; bab\,b^{*}aabab \right)^{*}
\]

\section*{ДКА}

\begin{center}
    \includegraphics[width=0.85\textwidth]{DFA.png} 
\end{center}


w1  = ababbababb

w2  = bb

w3  = $\varepsilon$

w4  = abb

w5  = bababa

w6  = b^5ababbababb

w7  = a

w8  = ababa

w9  = ab^3ababa

w10 = b^3ababa

w11 = babbababb

w12 = aaabbababa

w13 = babb

w14 = ab^4aabab

w15 = a^8b

w16 = a^9b

w17 = bbababa

w18 = b^2aabab

w19 = ab

w20 = b

w21 = abbababa

w22 = aabab

w23 = bab

w24 = abab



\begin{center}
\scriptsize
\setlength{\tabcolsep}{3pt} 
\renewcommand{\arraystretch}{1.5}
\begin{tabular}{l|*{24}{c}}
     & w1 & w2 & w3 & w4 & w5 & w6 & w7 & w8 & w9 & w10 & w11 & w12 & w13 & w14 & w15 & w16 & w17 & w18 & w19 & w20 & w21 & w22 & w23 & w24 \\ \hline
q6  & \textbf{+} & - & - & - & - & \textbf{+} & - & - & - & - & - & - & - & - & - & - & - & - & - & - & - & - & - & - \\
q12 & - & \textbf{+} & - & - & - & - & - & - & - & - & - & - & - & - & - & \textbf{+} & - & - & \textbf{+} & - & - & - & - & \textbf{+} \\
q14 & - & - & \textbf{+} & - & - & - & - & - & - & - & - & - & - & - & - & \textbf{+} & - & - & \textbf{+} & - & - & - & - & \textbf{+} \\
q11 & - & - & - & \textbf{+} & - & - & - & - & - & - & - & - & - & - & \textbf{+} & - & - & - & - & - & - & \textbf{+} & - & - \\
q0  & - & - & - & - & \textbf{+} & - & - & - & - & - & - & \textbf{+} & - & - & - & - & - & - & - & - & - & - & - & - \\
q5  & - & - & - & - & - & \textbf{+} & - & - & - & - & - & \textbf{+} & - & - & - & - & - & - & - & - & \textbf{+} & - & - & - \\
q13 & - & - & - & - & - & - & \textbf{+} & - & - & - & - & - & - & - & \textbf{+} & - & - & - & - & \textbf{+} & - & \textbf{+} & \textbf{+} & - \\
q1  & - & - & - & - & - & - & - & \textbf{+} & - & - & - & - & - & - & - & - & - & - & - & - & - & - & - & - \\
q2  & - & - & - & - & - & - & - & - & \textbf{+} & - & - & - & - & - & - & - & - & - & - & - & - & - & - & - \\
q3  & - & - & - & - & - & - & - & - & - & \textbf{+} & - & - & - & - & - & - & - & - & - & - & - & - & - & - \\
q7  & - & - & - & - & - & - & - & - & - & - & \textbf{+} & - & - & - & - & - & - & - & - & - & - & - & - & - \\
q9  & - & - & - & - & - & - & - & - & - & - & - & \textbf{+} & - & - & - & - & \textbf{+} & - & - & - & \textbf{+} & - & - & - \\
q10 & - & - & - & - & - & - & - & - & - & - & - & - & \textbf{+} & - & - & - & - & - & - & - & - & - & - & - \\
q15 & - & - & - & - & - & - & - & - & - & - & - & - & - & \textbf{+} & - & - & - & - & - & - & - & - & - & - \\
q22 & - & - & - & - & - & - & - & - & - & - & - & - & - & - & \textbf{+} & - & - & - & - & \textbf{+} & - & \textbf{+} & \textbf{+} & - \\
q23 & - & - & - & - & - & - & - & - & - & - & - & - & - & - & - & \textbf{+} & - & - & \textbf{+} & - & - & - & - & \textbf{+} \\
q4  & - & - & - & - & - & - & - & - & - & - & - & - & - & - & - & - & \textbf{+} & - & - & - & - & - & - & - \\
q16 & - & - & - & - & - & - & - & - & - & - & - & - & - & - & - & - & - & \textbf{+} & - & - & - & - & - & - \\
q20 & - & - & - & - & - & - & - & - & - & - & - & - & - & - & - & - & - & - & \textbf{+} & - & - & - & - & \textbf{+} \\
q21 & - & - & - & - & - & - & - & - & - & - & - & - & - & - & - & - & - & - & - & \textbf{+} & - & - & \textbf{+} & - \\
q8  & - & - & - & - & - & - & - & - & - & - & - & - & - & - & - & - & - & - & - & - & \textbf{+} & - & - & - \\
q17 & - & - & - & - & - & - & - & - & - & - & - & - & - & - & - & - & - & \textbf{+} & - & - & - & \textbf{+} & - & - \\
q19 & - & - & - & - & - & - & - & - & - & - & - & - & - & - & - & - & - & - & - & - & - & - & \textbf{+} & - \\
q18 & - & - & - & - & - & - & - & - & - & - & - & - & - & - & - & - & - & - & - & - & - & - & - & \textbf{+} \\
\end{tabular}
\end{center}


\section*{НКА}
Поскольку матрица переходов практически имеет верхнетреугольный вид (картину нарушает лишь элемент на позиции 
(q17,w18)), можно утверждать, что полученный минимальный ДКА одновременно является и минимальным НКА. Тем не менее, я дополнительно построю НКА, который будет использоваться для фазз-тестирования.

\begin{center}
    \includegraphics[width=0.85\textwidth]{NFA.png} 
\end{center}

В моем случае таблицу для НКА строить не нужно, так как подходит таблица для ДКА.

\section*{ПКА}

\textbf{Инварианты регулярного выражения:}
\begin{enumerate}
    \item Строка состоит только из символов $a$ и $b$
    \item Строка содержит хотя бы одну из следующих подстрок: \texttt{baba}, \texttt{bab}, \texttt{aba}, \texttt{ab}, \texttt{ba}
    \item Длина строки не менее 6 символов
    \item Во всех альтернативах строки оканчиваются на $b$(во второй части на $ab$)
\end{enumerate}

Построю ПКА по 3 инварианту со строкой baba.

\begin{center}
\scriptsize
\setlength{\tabcolsep}{3pt} 
\renewcommand{\arraystretch}{1.7}
\begin{tabular}{l|*{24}{c}}
     & w1 & w2 & w3 & w4 & w5 & w6 & w7 & w8 & w9 & w10 & w11 & w12 & w13 & w14 & w15 & w16 & w17 & w18 & w19 & w20 & w21 & w22 & w23 & w24 \\ \hline
q6  & \textbf{+} & - & - & - & - & \textbf{+} & - & - & - & - & - & - & - & - & - & - & - & - & - & - & - & - & - & - \\
q12 & - & \textbf{+} & - & - & - & - & - & - & - & - & - & - & - & - & - & \textbf{+} & - & - & \textbf{+} & - & - & - & - & \textbf{+} \\
q14 & - & - & \textbf{+} & - & - & - & - & - & - & - & - & - & - & - & - & \textbf{+} & - & - & \textbf{+} & - & - & - & - & \textbf{+} \\
q11 & - & - & - & \textbf{+} & - & - & - & - & - & - & - & - & - & - & \textbf{+} & - & - & - & - & - & - & \textbf{+} & - & - \\
q0  & - & - & - & - & \textbf{+} & - & - & - & - & - & - & \textbf{+} & - & - & - & - & - & - & - & - & - & - & - & - \\
q5  & - & - & - & - & - & \textbf{+} & - & - & - & - & - & \textbf{+} & - & - & - & - & - & - & - & - & \textbf{+} & - & - & - \\
q13 & - & - & - & - & - & - & \textbf{+} & - & - & - & - & - & - & - & \textbf{+} & - & - & - & - & \textbf{+} & - & \textbf{+} & \textbf{+} & - \\
q1  & - & - & - & - & - & - & - & \textbf{+} & - & - & - & - & - & - & - & - & - & - & - & - & - & - & - & - \\
q2  & - & - & - & - & - & - & - & - & \textbf{+} & - & - & - & - & - & - & - & - & - & - & - & - & - & - & - \\
q3  & - & - & - & - & - & - & - & - & - & \textbf{+} & - & - & - & - & - & - & - & - & - & - & - & - & - & - \\
q7  & - & - & - & - & - & - & - & - & - & - & \textbf{+} & - & - & - & - & - & - & - & - & - & - & - & - & - \\
q9  & - & - & - & - & - & - & - & - & - & - & - & \textbf{+} & - & - & - & - & \textbf{+} & - & - & - & \textbf{+} & - & - & - \\
q10 & - & - & - & - & - & - & - & - & - & - & - & - & \textbf{+} & - & - & - & - & - & - & - & - & - & - & - \\
q15 & - & - & - & - & - & - & - & - & - & - & - & - & - & \textbf{+} & - & - & - & - & - & - & - & - & - & - \\
q22 & - & - & - & - & - & - & - & - & - & - & - & - & - & - & \textbf{+} & - & - & - & - & \textbf{+} & - & \textbf{+} & \textbf{+} & - \\
q23 & - & - & - & - & - & - & - & - & - & - & - & - & - & - & - & \textbf{+} & - & - & \textbf{+} & - & - & - & - & \textbf{+} \\
q4  & - & - & - & - & - & - & - & - & - & - & - & - & - & - & - & - & \textbf{+} & - & - & - & - & - & - & - \\
q16 & - & - & - & - & - & - & - & - & - & - & - & - & - & - & - & - & - & \textbf{+} & - & - & - & - & - & - \\
q20 & - & - & - & - & - & - & - & - & - & - & - & - & - & - & - & - & - & - & \textbf{+} & - & - & - & - & \textbf{+} \\
q21 & - & - & - & - & - & - & - & - & - & - & - & - & - & - & - & - & - & - & - & \textbf{+} & - & - & \textbf{+} & - \\
q8  & - & - & - & - & - & - & - & - & - & - & - & - & - & - & - & - & - & - & - & - & \textbf{+} & - & - & - \\
q17 & - & - & - & - & - & - & - & - & - & - & - & - & - & - & - & - & - & \textbf{+} & - & - & - & \textbf{+} & - & - \\
q19 & - & - & - & - & - & - & - & - & - & - & - & - & - & - & - & - & - & - & - & - & - & - & \textbf{+} & - \\
q18 & - & - & - & - & - & - & - & - & - & - & - & - & - & - & - & - & - & - & - & - & - & - & - & \textbf{+} \\ \hline
p0  & \textbf{+} & - & - & - & \textbf{+} & \textbf{+} & - & \textbf{+} & \textbf{+} & \textbf{+} & \textbf{+} & \textbf{+} & - & - & - & - & \textbf{+} & - & - & - & \textbf{+} & - & - & - \\
p1  & \textbf{+} & - & - & - & \textbf{+} & \textbf{+} & - & \textbf{+} & \textbf{+} & \textbf{+} & \textbf{+} & \textbf{+} & - & - & - & - & \textbf{+} & - & - & - & \textbf{+} & - & - & \textbf{+} \\
p2  & \textbf{+} & - & - & - & \textbf{+} & \textbf{+} & - & \textbf{+} & \textbf{+} & \textbf{+} & \textbf{+} & \textbf{+} & \textbf{+} & - & - & - & \textbf{+} & - & - & - & \textbf{+} & - & \textbf{+} & - \\
p3  & \textbf{+} & - & - & \textbf{+} & \textbf{+} & \textbf{+} & \textbf{+} & \textbf{+} & \textbf{+} & \textbf{+} & \textbf{+} & \textbf{+} & - & \textbf{+} & \textbf{+} & \textbf{+} & \textbf{+} & - & \textbf{+} & - & \textbf{+} & \textbf{+} & - & \textbf{+} \\
p4  & \textbf{+} & \textbf{+} & \textbf{+} & \textbf{+} & \textbf{+} & \textbf{+} & \textbf{+} & \textbf{+} & \textbf{+} & \textbf{+} & \textbf{+} & \textbf{+} & \textbf{+} & \textbf{+} & \textbf{+} & \textbf{+} & \textbf{+} & \textbf{+} & \textbf{+} & \textbf{+} & \textbf{+} & \textbf{+} & \textbf{+} & \textbf{+} \\
\end{tabular}
\end{center}

\begin{center}
    \includegraphics[width=1.02\textwidth]{AFA.png} 
\end{center}

\section*{Расширенное регулярное выражение}

\[
\underbrace{\left( aa^{*}ab \;\middle|\; bbabb \;\middle|\; ab\,b^{*}abab \right)^{*}}_{\text{Блок 1}}
\, 
\underbrace{baba}_{\text{Блок 2}}
\underbrace{(a \mid b)(a \mid b)}_{\text{Блок 3}}
\underbrace{\left( (aa)^{*}ab \;\middle|\; bab\,b^{*}aabab \right)^{*}}_{\text{Блок 4}}
\]

\[
\Downarrow
\]

\[
\underbrace{^{\wedge}\left( aa^{+}b \;\middle|\; bbabb \;\middle|\; ab^{+}abab \right)^{*}}_{\text{Блок 1}}
\, 
\underbrace{baba}_{\text{Блок 2}}
\underbrace{..}_{\text{Блок 3}}
\underbrace{\left( ((aa)^{*} \;\middle|\; bab^{+}aab)ab \right)^{*}\$}_{\text{Блок 4}}
\]


\textbf{1. $aa^{*}ab \;\to\; aa^{+}b$.}  
Перед $a^{*}$ уже стоит обязательная буква $a$, поэтому минимум один символ $a$ уже гарантирован.  
Отсюда $a a^{*} = a^{+}$.

\textbf{2. $ab\,b^{*}abab \;\to\; ab^{+}abab$.}  
Аналогично, перед $b^{*}$ есть обязательный $b$, значит минимум один $b$ уже есть.  
Поэтому $b b^{*} = b^{+}$.

\textbf{3. $(a \mid b)(a \mid b) \;\to\; ..$.}  
Каждый из вариантов $(a \mid b)$ означает «любой символ из алфавита $\{a,b\}$».  
Две такие конструкции подряд дают два произвольных символа, поэтому заменяются на ``..''.


\textbf{4. $bab\,b^{*}aabab \;\to\; bab^{+}aabab$.}  
Опять используется $b b^{*} = b^{+}$, так как перед звёздочкой есть обязательный $b$.

\textbf{5. }  
Обе альтернативы $(aa)^{*}ab$ и $bab^{+}aabab$ заканчиваются на ab.  
Поэтому общий хвост можно вынести: $Xab \mid Yab = (X \mid Y)ab$.

\end{document}

